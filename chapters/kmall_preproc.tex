\chapter{KMALL Stage}
\begin{comment}
Aquesta etapa decorreladora està enfocada a dades provinents de sondes submarines de l'empresa KONGSBERG.

La informació més interessant dels fitxers .kmall (ordre dels 100 MB) és la de les columnes d'aigua (ocupa el 99\% de la mida total del fitxer). El format d'aquests fitxers està basat en datagrames, la documentació dels quals és pública a la web de KONGSBERG (com que és un dOxygen potser ho pugem al servidor de DAPCOM, ho he de parlar amb en Jordi).

Com que FAPEC comprimeix per chunks s'ha de parar especial atenció a no partir un datagrama per la meitat o F al xat.
\end{comment}


\section{Introduction}
Explicar les estructures de dades del KMALL basant-me en el dOxygen de Kongsberg.

\section{Design}
On it. Les idees són: no trencar datagrames, comprimir només MWC i potser MRZ (amb Python he analitzat que és worth de comprimir i què no). L'algoritme de compressió es basarà en l'anterior fet pels fitxer .wcd de KONGSBERG.

\section{Implementation}
Implementació amb C de l'algoritme. El codi anirà als annexes (a poder ser confidencials).

\section{Results}
No ens precipitem.