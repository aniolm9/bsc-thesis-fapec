\chapter{Wave Stage}
\begin{comment}
Aquesta etapa decorreladora està enfocada a dades amb 2 canals (àudio stereo, I/Q). El desenvolupament dins de l'empresa està especialment enfocat per un client, però per motius de confidencialitat inclouré una versió genèrica que es podrà comparar amb el FLAC. L'etapa fa servir LPC i posteriorment envia els errors de predicció al codificador entròpic.

L'algoritme és el "típic" per LPC i la resolució del sistema lineal resultant del predictor lineal \parencite{PSAVC} es fa amb Levinson-Durbin \parencite{LevinsonDurbin}, que aprofitant l'estructura Toeplitz redueix la complexitat de $O(n^3)$ a $O(n^2)$.
\end{comment}

Earth Observation techniques have been suffering a growth both in terms of quality and quantity, which means that the amount of data produced have also increased drastically. In addition, remote sensing is carried out by satellites with low storage capacity and bandwidth \parencite{SANDAU20101}, so clearly data compression is key in this situation.

Between the vast amount of remote sensing techniques we find Radio Occultation \parencite{RO-GNSS}, which uses low orbit satellites to detect the changes produced by the atmosphere in a radio signal emitted by a GNSS satellite.

\begin{figure}[h!]
	\begin{center}
		\begin{tabular}{ @{} r @{} }
			\includegraphics[scale=0.42]{images/ro_schematic.jpg}\\
			\imagesource{Wikipedia user MPRennie, CC BY-SA 3.0, via Wikimedia Commons.}
		\end{tabular}
	\end{center}
	\vspace*{-0.7em}
	\caption{Illustration of radio occultation (RO).}
	\label{fig:ro_schematic}
\end{figure}


\section{Introduction to I/Q data}
\begin{comment}
Donar una base de les dades I/Q i justificar perquè són semblants a l'audio stereo.
\end{comment}


\section{Design}
\begin{comment}
Disseny del predictor i l'estructura de l'etapa. Justificació de per què s'utilitzen els LPC i no una altra cosa. La resolució de les equacions de Yule-Walker anirà aquí, però segurament en posaré la deducció en un annex.
\end{comment}

\section{Implementation}
\begin{comment}
Implementació amb C de l'algoritme. El codi anirà als annexes (a poder ser confidencials).
\end{comment}

\section{Results}
\begin{comment}
D'això ja tinc alguna cosa provada i amb la versió d'ara, amb la cançó Money de Pink Floyd quasi iguala el FLAC. Acceptant unes miques de pèrdues (jo no les distingeixo) el supera molt.
\end{comment}
