\chapter{Entropy Coding}
\begin{comment}
Comentaré alguns algoritmes típics de compressió com el Huffman, Rice-Golomb, codis aritmètics i Asymmetric numeral systems (Zstandard)
\end{comment}

Entropy coding is a lossless data compression scheme based on symbols probability. This field was first described by Claude E. Shannon in 1948 in his paper \textit{A Mathematical Theory of Communication} \parencite{Shannon1948}.

Giving an extensive theoretical explanation of Entropy coding would require an entire thesis, therefore in this chapter we will define the most basic equations in Information Theory and also describe a few well-known coding algorithms that will be relevant later.

\begin{figure}[h!]
	\begin{center}
		\begin{tabular}{ @{} c @{} }
			\includegraphics[scale=0.3]{images/Claude_Shannon_1776.jpg}\\
			\imagesource{DobriZheglov, CC BY-SA 4.0, via Wikimedia Commons.}
		\end{tabular}
	\end{center}
	\vspace*{-0.7em}
	\caption{Picture of Claude E. Shannon staring at a mouse.}
	\label{fig:shannon}
\end{figure}

\section{Huffman coding}
\begin{comment}
Òptim, prefix, FLAC, etc. Es compararà amb l'algoritme WAVE de FAPEC a través del rendiment de FLAC per àudio lossless.
\end{comment}

\section{Golomb coding}
\begin{comment}
Els codis de Rice (subconjunt dels codis de Golomb) van donar origen a FAPEC, així que és important comentar-los.
També són codis de prefix com el Huffman però poden no ser òptims.
\end{comment}

\section{Arithmetic coding}
\begin{comment}
Aquesta secció la faré servir per donar pas a l'Asymmetric numeral systems. Bàsicament els codis aritmètics són molt bons però són molt lents. Això ho compensa els ANS.
\end{comment}

\section{Asymmetric Numeral Systems}
\begin{comment}
Em centraré especialment en el Zstandard (Facebook). Tenen nivells de compressió similars als aritmètics però van MOLT ràpid (una mica més que FAPEC i tot).
\end{comment}