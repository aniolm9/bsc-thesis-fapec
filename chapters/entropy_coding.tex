\chapter{Entropy Coding}
\begin{comment}
Comentaré alguns algoritmes típics de compressió com el Huffman, Rice-Golomb, codis aritmètics i Asymmetric numeral systems (Zstandard)
\end{comment}

\section{Huffman coding}
\begin{comment}
Òptim, prefix, FLAC, etc. Es compararà amb l'algoritme WAVE de FAPEC a través del rendiment de FLAC per àudio lossless.
\end{comment}

\section{Golomb coding}
\begin{comment}
Els codis de Rice (subconjunt dels codis de Golomb) van donar origen a FAPEC, així que és important comentar-los.
També són codis de prefix com el Huffman però poden no ser òptims.
\end{comment}

\section{Arithmetic coding}
\begin{comment}
Aquesta secció la faré servir per donar pas a l'Asymmetric numeral systems. Bàsicament els codis aritmètics són molt bons però són molt lents. Això ho compensa els ANS.
\end{comment}

\section{Asymmetric Numeral Systems}
\begin{comment}
Em centraré especialment en el Zstandard (Facebook). Tenen nivells de compressió similars als aritmètics però van MOLT ràpid (una mica més que FAPEC i tot).
\end{comment}