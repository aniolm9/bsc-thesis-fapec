\begin{tikzpicture}[
        every node/.style={draw,minimum width=3cm,minimum height=1cm},
        every text node part/.style={align=center},
         node distance=2.6cm and 3.8cm, on grid,
        >={Stealth[length=2mm]},
        inici/.style={rounded rectangle,minimum height=1cm,fill=ForestGreen!20},
        parametres/.style={
            trapezium,
            shape border rotate=90,
            trapezium left angle=90, 
            trapezium right angle=80,
            minimum width=1cm,
            fill=yellow!30,
        },
        entrada/.style={tape,minimum width=1.5cm,fill=violet!20},
        rectangle/.style={minimum width=2.2cm,fill=blue!10},
        decisio/.style={diamond,minimum height=1.7cm,aspect=1.6,fill=red!10},
        coordenada/.style={draw=none,minimum width=0},
        bool/.style={draw=none,minimum width=0,minimum height=0},
    ]
    \scriptsize
    % Dibuixo primer els nodes i després les fletxes. Hi ha diverses maneres de fer-ho
    \node[inici,minimum width=4cm] (start) {Start};
    \node[entrada] (input) at (-1.3,2) {Input file};
    \node[parametres] (settings) at (1,2) {Settings};
    \node[decisio,below=of start] (D1) {Read bytes =\\buffer size?};
    \node[inici, right=of D1] (end) {End};
    \node[decisio,below=of D1] (D2) {Is MWC\\ datagram?};
    \node[rectangle,below=of D2] (A) {Entropy coder};
    \node[decisio, right=of D2] (D3) {Payload data?};
    \node[decisio, right=of D3] (D4) {Lossy?};
    \node[rectangle,below=of D4] (B)  {Divide samples};
    \node[rectangle,right= of D4] (C) {$E_{0,0} = S_{0,0}+128$};
    \node[rectangle,below=of C] (D) {$E_{i,0} = S_{i,0}-S_{i-1,0}$};
    \node[decisio,aspect=1.6,below=of D] (D5) {$i<\min(N_{S_j}, N_{S_{j-1}})$?};
    \node[rectangle,below=of D5] (E) {$E_{i,j} = S_{i,j}-S_{i,j-1}$};
    \node[rectangle,left=of E] (F) {$E_{i,j} = S_{i,j}-S_{i-1,j}$};
    \node[decisio,below=of E] (D6) {Phase?};
    \node[rectangle,below=of D6] (G) {$E_{i,j} = S_{i,j}-S_{i-1,j}$};

    % Ara les fletxes, aprofitant que he donat nom als nodes
    \draw[->] (start) -- node[coordenada,midway,name=P1] {} (D1);
    \draw[->] (D1) -- (end);
    \draw[->] (D1) -- (D2);
    \draw[->] (D2) -- node[coordenada,midway,name=P2] {} (A);
    \draw[->] (A.west) -- ++(-0.8,0) |- (P1.center);
    \draw[->] (D2) -- (D3);
    \draw[->] (D3) |- node[coordenada,pos=0.73,name=P3] {} (P2.center);
    \draw[->] (D3) -- (D4);
    \draw[->] (D4) -- (B);
    \draw[->] (D4) -- node[coordenada,midway,name=P4] {} (C);
    \draw[->] (B) -| (P4.center);
    \draw[->] (C) -- (D);
    \draw[->] (D) -- node[coordenada,midway,name=P5] {} (D5);
    \draw[->] (D5) -| (F);
    \draw[->] (D5) -- (E);
    \draw[->] (E) -- node[coordenada,midway,name=P6] {} (D6);
    \draw[->] (F) |- (P6.center);
    \draw[->] (D6) -| node[coordenada,pos=0.25,name=P7] {} (P3.center);
    \draw[->] (D6) -- (G);
    \draw[->] (G) -| (P7.center);

    \draw[->] (input) -- ++(0,-1.5);
    \draw[->] (settings) -- ++(0,-1.5);

    % Ara els True/False escampats aquí i allí
    \node[bool,anchor=south west] at (D1.east) {True};
    \node[bool,anchor=north east] at (D1.south) {False};

    \node[bool,anchor=south west] at (D2.east) {True};
    \node[bool,anchor=north east] at (D2.south) {False};

    \node[bool,anchor=south west] at (D3.east) {True};
    \node[bool,anchor=north east] at (D3.south) {False};

    \node[bool,anchor=south west] at (D4.east) {False};
    \node[bool,anchor=north east] at (D4.south) {True};

    \node[bool,anchor=south east] at (D5.west) {False};
    \node[bool,anchor=north east] at (D5.south) {True};

    \node[bool,anchor=south east] at (D6.west) {False};
    \node[bool,anchor=north east] at (D6.south) {True};
\end{tikzpicture}
