\begin{tikzpicture}[
        every node/.style={draw,minimum width=3cm,minimum height=1cm},
        every text node part/.style={align=center},
         node distance=2.6cm and 3.8cm, on grid,
        >={Stealth[length=2mm]},
        inici/.style={rounded rectangle,minimum height=1cm,fill=ForestGreen!20},
        parametres/.style={
            trapezium,
            shape border rotate=90,
            trapezium left angle=90, 
            trapezium right angle=80,
            minimum width=1cm,
            fill=yellow!30,
        },
        entrada/.style={tape,minimum width=1.5cm,fill=violet!20},
        rectangle/.style={minimum width=2.2cm,fill=blue!10},
        decisio/.style={diamond,minimum height=1.4cm,aspect=1.6,fill=red!10},
        coordenada/.style={draw=none,minimum width=0},
        bool/.style={draw=none,minimum width=0,minimum height=0},
    ]
    \scriptsize
    % Dibuixo primer els nodes i després les fletxes. Hi ha diverses maneres de fer-ho
    \node[inici,minimum width=4cm] (start) {Start};
    \node[entrada] (input) at (-1.3,2) {Input file};
    \node[parametres] (settings) at (1,2) {Settings};
    \node[decisio,below=of start] (D1) {More periods?};
    \node[inici, below=of D1] (end) {End};
	\node[rectangle, right=of D1] (samples) {Extract samples};
	\node[decisio,below=of samples] (D2) {More channels?};
	\node[decisio,below=of D2] (D3) {Lossy?};
	\node[rectangle, right=of D3] (divide) {Divide samples};
	\node[decisio,below=of D3] (D4) {Coupling?};
	\node[decisio,right=of D4] (D5) {First channel?};
	\node[rectangle, below=of D4] (autocorr) {Find autocorrelation};
	\node[rectangle, below=of autocorr] (levinson) {Levinson-Durbin};
	\node[rectangle, below=of levinson] (estimations) {Find estimation errors};
	\node[rectangle, below=of estimations] (coder) {Entropy coder};

	% Dibuixo les fletxes
	\draw[->] (input) -- ++(0,-1.5);
    \draw[->] (settings) -- ++(0,-1.5);
	\draw[->] (start) -- node[coordenada,midway,name=P1] {} (D1);
	\draw[->] (D1) -- (end);
    \draw[->] (D1) -- (samples);
    \draw[->] (samples) -- node[coordenada,midway,name=P2] {} (D2);
    \draw[->] (D2) -- (D3);
	\draw[->] (D3) -- (divide);
	\draw[->] (D3) -- node[coordenada,midway,name=P3] {} (D4);
	\draw[->] (divide) |- (P3.center);
	\draw[->] (D4) -- (D5);
	\draw[->] (D4) -- node[coordenada,midway,name=P4] {} (autocorr);
	\draw[->] (D5) |- (P4.center);
	\draw[->] (autocorr) -- node[coordenada,midway,name=P5] {} (levinson);
	\draw[->] (D5.east) -- ++(0.8,0) |- (P5.center);
	\draw[->] (levinson) -- (estimations);
	\draw[->] (estimations) -- (coder);
	\draw[->] (coder.west) -- ++(-0.8,0) |- (P2.center);
	\draw[->] (D2.east) -- ++(0.8,0) |- (P1.center);

	% Ara els True/False escampats aquí i allí
    \node[bool,anchor=south west] at (D1.east) {True};
    \node[bool,anchor=north east] at (D1.south) {False};

    \node[bool,anchor=south west] at (D2.east) {False};
    \node[bool,anchor=north east] at (D2.south) {True};

    \node[bool,anchor=south west] at (D3.east) {True};
    \node[bool,anchor=north east] at (D3.south) {False};

    \node[bool,anchor=south west] at (D4.east) {True};
    \node[bool,anchor=north east] at (D4.south) {False};

    \node[bool,anchor=south west] at (D5.east) {False};
    \node[bool,anchor=north east] at (D5.south) {True};
\end{tikzpicture}