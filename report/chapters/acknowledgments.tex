\chapter*{Agraïments}
Ai, els agraïments. Els agraïments són, amb diferència, la part més difícil de tot el treball. Amb això no pretenc pas treure valor a aquest treball, només faltaria, però els agraïments els llegeix tothom: tant qui busca el contingut del treball com qui només vol xafardejar què has fet. Per això són importants, i també perquè no pots deixar-te ningú. Tanmateix, on traces la línia? En Jaime, que em va servir una cervesa quan estava fart d'escriure, ha d'anar als agraïments? No ho sé, però decidir això és molt feixuc, i després de quatre mesos de treballar, estudiar i fer aquest treball doncs el meu electroencefalograma és paral·lel a l'eix de les abscisses. Així que, a continuació, donaré les gràcies a les persones que han estat més a prop i que he empipat més aquests mesos.

Primer de tot vull donar les gràcies als meus tres directors: en Ferran de Cabrera, en Jordi Portell i en Jaume Riba. A en Jordi per donar-me la possibilitat de fer aquest treball, a en Jaume i a en Ferran per totes les observacions i aportacions teòriques que han fet, i a tots tres per les revisions, reunions i dubtes que m'han resolt. Gràcies.

A part d'ells, també mereixen ser en aquests agraïments en Marc Vilà, que ha assistit a moltes de les reunions; en David Amblàs, sense el qual molts resultats no haurien estat possibles; i l'Orestes Mas, que m'ha ajudat amb molts detalls del LaTeX i el disseny del document.

Deixant de banda l'àmbit acadèmic i professional he de donar les gràcies a molts amics que han estat allà per fer una cervesa, jugar al Counter Strike, anar al cine o per, simplement, parlar i desconnectar.

Finalment, pels que no sou aquí i creieu que hauríeu de ser-hi, recordeu que sempre sereu al meu cor. I ara, Jaime, posa'm una altra gerra.
