\chapter{Wave Stage}
\begin{comment}
Aquesta etapa decorreladora està enfocada a dades amb 2 canals (àudio stereo, I/Q). El desenvolupament dins de l'empresa està especialment enfocat per un client, però per motius de confidencialitat inclouré una versió genèrica que es podrà comparar amb el FLAC. L'etapa fa servir LPC i posteriorment envia els errors de predicció al codificador entròpic.

L'algoritme és el "típic" per LPC i la resolució del sistema lineal resultant del predictor lineal \parencite{PSAVC} es fa amb Levinson-Durbin \parencite{LevinsonDurbin}, que aprofitant l'estructura Toeplitz redueix la complexitat de $O(n^3)$ a $O(n^2)$.
\end{comment}

Earth Observation techniques have been suffering a growth both in terms of quality and quantity, which means that the amount of data produced have also increased drastically. In addition, remote sensing is carried out by satellites with low storage capacity and bandwidth \parencite{SANDAU20101}, so clearly data compression is key in this situation.

Between the vast amount of remote sensing techniques we find Radio Occultation \parencite{RO-GNSS}, which uses low orbit satellites to detect the changes produced by the atmosphere in a radio signal emitted by a \acrshort{gnss} satellite.

\begin{figure}[h!]
	\begin{center}
		\begin{tabular}{ @{} c @{} }
			\includegraphics[scale=0.42]{images/ro_schematic.jpg}\\
			\imagesource{Wikipedia user MPRennie, CC BY-SA 3.0, via Wikimedia Commons.}
		\end{tabular}
	\end{center}
	\vspace*{-0.7em}
	\caption{Illustration of radio occultation (RO).}
	\label{fig:ro_schematic}
\end{figure}


\section{Introduction to IQ data}
\begin{comment}
Donar una base de les dades I/Q i justificar perquè són semblants a l'audio stereo.
\end{comment}

In the introduction of this chapter we stated that the present stage will be focused on \acrshort{iq} data, that is, the discrete time samples of a \acrshort{rf} \acrshort{iq} signal.

From Communication Theory we know that any pass-band signal $s(t)$ can be expressed as:
\begin{equation}
s(t) = i_s(t) \cdot cos(2\pi f_0 t) - q_s(t) \cdot sin(2\pi f_0 t)
\end{equation}

Where:
\begin{description}
	\item $i_s(t) \equiv$ In-phase component of the RF signal.
	\item $q_s(t) \equiv$ Quadrature component of the RF signal.
	\item $f_0 \equiv$ Carrier frequency of the RF signal.
\end{description}

For simplicity, we may work with the equivalent base-band signal $b_s(t)$:
\begin{equation}
	b_s(t) = i_s(t) + j q_s(t)
\end{equation}

Hence, the data to be compressed are the discrete time samples of the signals $i_s(t)$ and $q_s(t)$. In other words, our aim is to compress time series data separated in two channels.

For further information on \acrshort{iq} signals the reader may check references \parencite{IQintro}, \parencite{carlson2010communication} and \parencite{ICOM}.

To conclude this section, we shall provide a qualitative justification on why we can use audio files to test \textit{Wave} and why we can compare it with the \acrshort{flac} standard.

\begin{comment}
Bàsicament les dades IQ i l'àudio stereo són sèries temporals de dos canals. El FLAC utilitza codis de Rice-Golomb que funcionen bé en sèrie temporals (veure article original de Golomb) i FAPEC està inspirat en ells (veure article d'en Jordi). També, com a justificació inversa, hi ha el paper dels japonesos que comprimeixen IQ amb FLAC.
\end{comment}

\section{Design}
\begin{comment}
Disseny del predictor i l'estructura de l'etapa. Justificació de per què s'utilitzen els LPC i no una altra cosa. La resolució de les equacions de Yule-Walker anirà aquí, però segurament en posaré la deducció en un annex.
\end{comment}

In the previous section we justified why \acrshort{flac} should have a good performance with \acrshort{iq} data. Taking this into account, the proposed algorithm should have a structure similar to \acrshort{flac}'s but using \acrshort{fapec} as the entropy coder. This should translate into a compressor faster than \acrshort{flac} and with a similar compression ratio.

\section{Implementation}
\begin{comment}
Implementació amb C de l'algoritme. El codi anirà als annexes (a poder ser confidencials).
\end{comment}

\section{Results}
\begin{comment}
D'això ja tinc alguna cosa provada i amb la versió d'ara, amb la cançó Money de Pink Floyd quasi iguala el FLAC. Acceptant unes miques de pèrdues (jo no les distingeixo) el supera molt.
\end{comment}
