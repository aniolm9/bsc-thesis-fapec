\chapter*{Abstract}
Recent technological advances have resulten in massive data collection, also in hazard environments such as the sea floor or outer space. In these situations, both computer power and downlink bandwidth are quite limited. For this reason, data compression becomes a basic technology.

In this thesis we propose two new stages for the \acrshort{fapec} data compressor: the first one is specially designed to decorrelate data from radiofrequency signals (\acrshort{rf}); and the second one is intended to preprocess multibeam water column data collected by Kongsberg Maritime echosounders.

In order to evaluate the algorithms we have developed, we propose using negentropy, a measure invariant to linear maps which gives us the distance from our data to gaussianity. Finally, we compare the presented stages with other well-known compression algorithm such as \acrshort{flac} (audio compression) and GZIP (general purpose compression).

\chapter*{Resum}
Els avenços tecnològics dels darrers anys han portat a recollir grans quantitats de dades, també en entorns violents com el fons marí o l’espai. En aquests ambients, tant la capacitat dels ordinadors de bord com l’amplada de banda del canal de comunicacions sol ser molt limitat. És per això que la compressió de dades resulta ser fonamental.

En aquest treball proposem dues noves etapes de preprocessat pel compressor \acrshort{fapec}: la primera està dissenyada per decorrelar dades provinents de senyals de radiofreqüència (\acrshort{rf}); i la segona per dades de columna d’aigua recollides per ecosondes submarines de Kongsberg Maritime.

Per tal d’avaluar el comportament dels algoritmes desenvolupats, proposem utilitzar la neguentropia, una mesura invariant a aplicacions lineals que ens aporta la distància d’un conjunt de dades respecte la gaussianitat. Finalment, comparem les etapes amb altres algoritmes com el \acrshort{flac} (compressor d’àudio) o el GZIP (compressor de propòsit general).