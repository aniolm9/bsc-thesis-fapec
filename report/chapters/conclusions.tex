\chapter{Conclusions and future work}
This project had a very clear objective since the beginning: develop two preprocessing stages for \acrfull{rf} and KMALL \acrshort{mwc} data. The process to reach this goal was not straightforward, because in order to propose an algorithm one needs to know well the data structure. However, understanding the data structure is not enough, as one must also understand the \acrshort{fapec} framework, and it has truly been one of the hardest parts of this project.

In addition to the two stages, we also wanted to propose a universal metric to evaluate them. We finally decided to use negentropy, but arriving there required several meetings and many hours to develop the basic theoretical framework, which we decided to include in chapters \ref{ch:entropy_coding} and \ref{ch:new_stages}. Apart from the formal definition of negentropy (see section \ref{sec:negentropy}), we also present Equation \ref{eq:negentropy_approx} to approximate it from data, which resulted to be very useful thanks to its consistency and its low computational complexity.

In sections \ref{sec:wave_results} and \ref{sec:kmall_results} we have evaluated the Wave and the KMALL stages, respectively. In the former, we have used a classical linear predictor to estimate the input samples and then the prediction errors have been calculated. Remember from section \ref{sec:fapec_overview} that the prediction errors are the values sent to the \acrshort{fapec} entropy coder, hence we have compared the negentropy of the input file samples and the prediction errors. In Table \ref{tab:negentropies_wave} we can see that negentropy increases for all the files in our dataset.

On the other hand, the achieved compression ratios are a bit worse than those of \acrshort{flac}. However, we are using a filter order of 10 which is quite high, and even in this situation \acrshort{fapec} is much faster and its speed compensates the ratios, as can be seen in Table \ref{tab:audio_compare}.

In the KMALL stage we have developed a tailored algorithm for the \texttt{.kmwcd} files from Kongsberg Maritime. The algorithm takes advantage of the high correlation between samples in the same column (see Figure \ref{fig:wc_amplitude}) to estimate a sample from the sample in the previous row but the same column, when possible (see section \ref{sec:kmall_design} for a more concise explanation). Following this approach, negentropy increases by about two orders of magnitude, as can be seen in Table \ref{tab:kmall_compare}. At this point it is worth to point out that trying to run \acrshort{fapec} on \texttt{.kmwcd} files without any kind of preprocessing results in no compression at all, result that matches with the zero negentropy in the second column of the previous table.

The last important result is that \acrshort{fapec} with the KMALL stage performs better both in term of process time and compression ratio than GZIP (see figure \ref{fig:kmall_compare}). In average, \acrshort{fapec} has a compression ratio 1.4 times better and is 2.5 times faster. Finally, note that the KMALL stage is only compared with GZIP because we were not able to find any specific algorithms for \texttt{.kmwcd} files, if they exist at all.

Our final words are that the results are satisfactory and meet the specifications, although some stages are open for future work and improvements. In the following, we list different ideas that can be explored beyond this work:
\begin{itemize}
	\item In the KMALL stage, analyze the entropy of the original samples per beam. This procedure is similar to the one in \parencite{MBESComp} and could help to detect where the entropy is lower and apply different techniques on different zones.
	\item Explore other audio encoding algorithms such as \acrshort{tak} in order to take some ideas and improve the Wave stage.
	\item Improve the KMALL stage with a new algorithm to compress MRZ datagrams, which take around a 90\% of \texttt{.kmall} files and contain the most commonly used information.
\end{itemize}

Besides improving the stages here proposed, another interesting line of work is to find a relation between the negentropy and the compression ratio achieved by \acrshort{fapec}. As already observed in section \ref{sec:wave_results}, the negentropy of \texttt{04 - Pink Floyd - Time.wav} is 2.42 times better than that of \texttt{06 - Manel - Els entusiasmats.wav}, but the compression ratios are 0.64 and 0.68, which clearly do not have a relation of 2.42. This research will hopefully conclude with the publication of a paper.

Since the beginning this project was a way to open the door to different formats which \acrshort{fapec} did not support. After this first proposal, the proposed stages will be implemented and used in real situations with real data, revealing the high applicability of this work.
